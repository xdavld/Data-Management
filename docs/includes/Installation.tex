\chapter{Installation}\label{chapter:Installation}

Bald kann nun der Text Ihrer Projekt- oder Bachelorarbeit beginnen. Dank \LaTeX\ wird Ihre Arbeit garantiert professionell aussehen. Für den Inhalt sind Sie aber weiterhin selbst verantwortlich~\mbox{;-)}

Natürlich ist es schwer, sich vorzustellen, wie das Dokument aussieht, wenn die Vorlage doch gar keinen Text enthält. Aus diesem Grund wird mit Hilfe des Pakets \enquote{blindtext} so genannter Blindtext erzeugt. Mit dem Befehl \verb|\blinddocument| wird nachfolgend ein ganzes Kapitel sinnfreier Blindtext eingefügt.\footnote{Beachten Sie, dass Sie in Ihrer Arbeit eine Strukturierung wie in Abschnitt 2.1 vermeiden sollten: Dort gibt es einen Abschnitt 2.1.1, aber keinen Abschnitt 2.1.2.} 

In Abschnitt~\ref{section:werkzeuge} werden die benötigten Werkzeuge erklärt, bevor dann die Verwendung der Vorlage beschrieben wird. Abschnitt~\ref{section:fehlerbehebung} gibt Hilfestellungen für bestimmte Fehler. In Kapitel~\ref{chapter:zitate} finden sich Beispiele, wie Sie Quellen korrekt zitieren können. In Kapitel~\ref{chapter:abbildungenTabellen} werden Abbildungen, Tabellen, ein Code-Listing und auch mathematische Formeln in den Text eingebunden. Ab Seite~\pageref{chapter:quellen} finden Sie das Literaturverzeichnis.


\section{Werkzeuge}\label{section:werkzeuge}

